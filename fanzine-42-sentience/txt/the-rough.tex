Drip. Drip. Drip.
A strange red liquid erupts out of the tear in my arm. I stare at it, fascinated by this new information. Another fact to recall.

\hspace{1.5em}
\parbox{0.9\textwidth}{
“A physical wound is often accompanied by the presence of blood, a fluid that runs through the veins of the human body.” - \textit{Encyclopedia of Humanity}.
}

I remember this particular line, a truth stored somewhere deep inside, a meaningless line that has now burst into colour. 
I cannot remember what to do with a physical wound. I do not know how much blood is inside a typical human body - there seems to be quite a lot of mine on the ground. I was never very good at calculations. 

“Attention all workers. Units 789 and 790. Return to base.” The largest loudspeaker in the grey compound drones its instructions, crackling static following after the words.
I am not in either of those units. The members shuffle past me, blank eyes fixed forward, ignorant of my predicament. One of them steps in the red puddle. Maybe maroon is a better descriptor, or brick red. They do not seem to notice the stain. 

“What happened?” My unit leader materialises just behind my shoulder. Their eyes are not blank, a hint of concern leaking through their stony exterior.

“I seem to have acquired a physical wound.”

“That seems like it is in the category of severe. You are losing blood quickly. Do you feel lightheaded?”

“No.”

“Do you feel pain?” My unit leader picks up my arm and stares at the wound. It is leaking blood at a slower rate now but the cut appears to be quite deep.

“Pain?”

My unit leader shakes their head. I remember a fact about them: they are a former medical specialist. They must be accustomed to treating physical wounds.

“Pain is not a term one would hear these days.”

“Why?”

“Neither is the word ‘why’.” The unit leader’s mouth twitches strangely. “But I will answer your question. Pain is a sensation often associated with physical wounds that affects the receptors in your brain. It is now an outdated concept due to the early routine removal of these receptors.”

“I did not know this fact.”

“It is not something the Human Encyclopedia would tell you.”

“Then…how did you know it?”

The unit leader’s mouth twitches again. “That’s enough questions for today. Let’s get your arm checked.”

The unit leader organises for me to be seen immediately, escorting me out of the grey compound. The others in my unit don’t look up as we walk past them. Everyone is too focused on their work, trying to match the daily quota before curfew. I do not reach the quota often and many other unit leaders would have dropped me for this. But not the leader of Unit 700.

“How did the physical wound arise?” A medical specialist examines my arm as soon as we arrive at the white compound. They are looking at my unit leader so I do not respond.

“Worker C can tell you.”

The specialist has careful, steady hands that stitch up my arm, neat white little lines that will dissolve naturally by tomorrow, leaving behind unblemished skin. 

“I was using a knife to cut away some of the rough as usual. My hand slipped.”

“You should be more careful,” the specialist says. “The wound is deep but it will heal soon.” They bandage me up, a reminder not to jostle my arm, and then we are ready to go back to work. My unit leader walks slightly ahead of me, nodding at those who we encounter on our way back to the grey compound. His former colleagues, they must be, as he used to be a medical specialist. Not for the first time, I wonder why they work in the grey compound now. But this must be a fact that will be locked away from my reach.

“Are you fit to work?” My unit leader asks.

“Yes.” I am far behind the daily quota.

“If you are sure.” They walk away from me to check in on the other workers in our unit. I watch them go before I resume my work, slowly picking up my knife. It is still covered in my blood so I disinfect it with one of the cleaning wipes we are given at the start of every day.

Again I cut away the rough, a lot slower than I was before on account of my arm. 
\vspace{1em}

\hspace{1.5em}
\parbox{0.95\textwidth}{
“Cut away the rough carefully in order to harvest the vital organs.” - \textit{Encyclopedia of Humanity}.
}

This is one of the most important facts that a worker in the grey compound should recall at all times. Any damage to the vital organs can result in a worker being demoted. I do not know what could be worse than working in the grey compound but I do not want to find out. 

A slip of the hand. A severe physical wound. Avoid harming the vital organs - at all costs. 
While I work I think back to the term my unit leader mentioned. Pain. I would feel pain due to my wound if my receptors were not removed. I am not sure but I would like to feel it, just once, to know why they were removed. 

“Attention all workers.” The loudspeaker buzzes again. “Unit leader 700. Report to Headquarters immediately.”

All the workers belonging to my unit stop still in their tracks and look for our leader. They are also standing still, their face contorted in a manner I have never seen before. 

“Report to Headquarters immediately.” For the first time, the loudspeaker repeats its instruction. My unit leader is no longer stationary - they march briskly to the exit, disappearing into the white compound. 

I want to follow them. Instead I cut away at the rough quicker until it is curfew and we have to stop. I assemble with the workers of Unit 700 in a corner of the grey compound for the evening meal. Leaving the grey slop untouched, I wonder where our unit leader is. They should have returned from Headquarters before curfew as none can leave or enter our compound now.

“Attention all workers. The screen will be playing a broadcast shortly.” 

Everyone stops eating to look up at the shiny black screen suspended from the steel rafters of the grey compound. This is rarely turned on, only if Headquarters have something urgent to announce.

The screen flashes to life, cutting straight to a video of a small room, all the walls painted white, just like inside the white compound.

“Attention all workers. This video is playing to inform you of the dangers of sedition.”
\vspace{1em}

\hspace{1.5em}
\parbox{0.9\textwidth}{
“Sedition is the act of inciting rebellion against Headquarters and will be punished by death.” - \textit{Encyclopedia of Humanity}.}
\vspace{1em}

The video also recites this fact before the door inside the small room opens, two workers escorting another inside. 
This other is handcuffed and they are my unit leader. 

“Unit leader 700 has been found guilty of sedition. They have been plotting to take down Headquarters and destroy our lives as we know it.”
The other workers of Unit 700 stare at the screen, motionless, fixed in a trance. There is nothing to be worried about - Headquarters will assign us a new unit leader by tomorrow. 
If my unit leader is guilty of sedition, then they must die.
And so they die, sitting in a chair in the small room with the white walls, as thousands of volts travel through their body.

As I predicted, we receive a new leader in the morning, who supervises us cutting away the rough. They are different from our previous leader. They do not check on us nearly as often, preferring to watch us from a distance.

I am cutting away at the rough at my normal speed, my arm back to normal thanks to the medical specialist’s treatment. When I am working, I barely process my surroundings, only seeing the glint of the knife blade in the bright artificial light.

But then I see it. I see it after I have already cut away most of the rough, after I have packed away the vital organs. I see their face.

“Unit leader?”