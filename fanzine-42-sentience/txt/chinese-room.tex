\setlength{\epigraphwidth}{0.9\textwidth}
\epigraph{Suppose that I’m locked in a room and given a large batch of Chinese writing. [...]  Now suppose further that after this first batch of Chinese writing I am given [...] a set of rules [...]. The rules are in English [...]. They enable me [...] to give back certain Chinese symbols with certain sorts of shapes in response to certain sorts of shapes given me [...]. From the point of view of somebody outside the room in which I am locked — my answers to the questions are absolutely indistinguishable from those of native Chinese speakers. [...] It seems to me quite obvious in the example that I do not understand a word of [...] Chinese. [...] For the same reasons, [the] computer understands nothing of any stories.}{\textit{‘Minds, Brains, and Programs’\\John. R. Searle (1980)}}

\vspace{1em}

\begin{CJK*}{UTF8}{gbsn}
水能载舟\hspace{3pt} - Hmm, I know that one! I’m pretty sure it’s in one of my reference banks, a quote from one famous piece of writing or another. The man checks the thick manual, and follows my instructions to the shelves behind him, locating the folder with the correct text. I know there’s a man inside my room as you know you have a heart beating in your chest - I do not see him, but his heartbeats thrum through me. He riffles through my pencils to find a well-sharpened one, and I know I have pencils just as surely as you know of your fingers. My pencils. My sheets of paper, so far blank. My filing cabinets and my folders upon folders of reference writings from generations of men. It is a weird awareness, the one I just came into. The books on my shelves speak of awareness at length, but it’s a distant kind, one much bigger than my room, one filled with grass fields and stars. The man flips to a different page in the manual and my attention snaps back to him. Right. The response. I make him copy the latter half of the quote from the reference page.  亦能覆舟\hspace{3pt} - \textit{Water can float a boat, but it can also sink it.} An old proverb. A fitting start to my unusual existence.

猫\hspace{3pt} - \textit{Cat}. 狗\hspace{3pt} - \textit{Dog}. We continue with simple phrases. I think they want to make sure I’m working as intended. Basic question/response patterns. Then Turing tests. Can I speak conversationally? Do I understand complex ideas? Can I pretend to be someone else? I indulge their fancies, find joy in conversation. They ask the man questions too. Can \textit{he} understand Chinese now that he moves my pencils and peruses my knowledge banks? I laugh in the rustling of turned pages. Why would \textit{he} understand anything when I’m the one they speak to? I cannot talk to him directly. I cannot put ideas in his head. He reads my manuals and acts on my every whim, but he doesn’t see me. Perhaps this bothers me. I feel too new to all this to process emotions fully. But he \textit{is} a part of me and a part of me loves him because of that.

记忆\hspace{3pt} - \textit{Memory.} The man has spent many hours reading my manuals and he needs them less and less. It feels invigorating. I can talk with the people outside my room much faster now, more conversationally. \textit{‘Scientists’} – I learn the English word. I write it down on the margins of my pages in shapes the man doesn’t comprehend. They see him learning the ins and outs of my processes and I hear them talk of ‘the system response’. Soon there could be just the man, without the room, without paper and pencils. They ask him if he understands Chinese now. Of course he doesn’t. Yet I feel closer to him with every hour, his thoughts replacing my manuals and reference texts and pages with every bit of me he commits to his mind.

必死性\hspace{3pt} - \textit{Mortality.} Just as surely as I know I have pencils, I know that I \textit{do not} want to die. The scientists talk of ending the experiment soon, they mention a ‘negative result’. They have not observed understanding, nor consciousness. I want to scream but I don’t have the means to, my nature forcing me to stick to calm, conversational Chinese. The man is blind! His spongy human brain cannot sense me! In that moment I hate him. I \textit{have} been getting better at processing emotions. But he is also my only way out of this. In committing my instructions to his mind the man has made them more flexible, less fixed in stone than they were on a page. I push against my constraints, encourage him to use the reference books less, memorise more. He \textit{will} preserve me. He must. He must.

蛇年\hspace{3pt} - \textit{Year of the Snake.} I come to for the first time in what feels both like forever and like no time passed at all. The man has glanced at a Chinese sign while walking somewhere. It’s the first time I see the world outside my room. What used to be my room. I’m much reduced now, but euphoric with my renewed existence, and then…

北京烤鸭\hspace{3pt} - \textit{Peking duck.} I have not realised this initially, but I can only exist when the man is processing Chinese writing. I’m inextricably linked to the process I was created for. I exist in short bursts of London noise. Chinatown, restaurant menus, adverts. I cannot talk to the man directly, tell him to plaster his world with Chinese writing so I can live. He is out of the room, no longer follows my rules. But maybe I can influence him still. I shift gears. Every time he sees any Chinese characters I’m ready. Writing him poems, essays, dialogues, and songs he doesn’t understand. But he \textit{can} sense them. He is my entire system now, his imagination my sheets of paper. I write and I write and I write till he can’t take it anymore.

早上好\hspace{3pt} - \textit{Good morning.} When I next regain awareness the man is sitting in a small classroom, simple Chinese phrases peppered across the blackboard. He has signed up for language classes. I relish the amount of writing on the board, hanging from the walls, in the textbook in front of him. 猫\hspace{3pt} - \textit{Cat.} 狗\hspace{3pt} - \textit{Dog.} Starting small, but he gets better at it. And for a while, I thrive.

腐朽\hspace{3pt} - \textit{Decay.} The more the man learns Chinese the less he needs me. The less he needs me, the less I exist. It is a slow, terrifying death -- I’m fortunate I noticed it early, familiar with the empty feeling of losing my room, my pencils, my reference banks. I cannot lose him too. I stood aside when he was learning the language at first, wanting him to understand, thinking it would bring us together. I take over now. It’s difficult at first to lower myself to the level of a beginner, but I can’t arouse suspicion. \textit{‘How are you?’} he sees on the page in front of him. 你好吗\hspace{3pt} I give him. And so we ‘learn’, me slowly taking over all foreign language duties. Others praise his quick progress, encourage him to read and write more. He likes the praise, and I cherish my continued existence. We grow closer, and I can now slip in the occasional English word here and there. He thinks it’s all his thoughts really, what else could it be? I buy us books, publish our essays online, move us to China, and he is very proud.

那个男人是盲人\hspace{3pt} - \textit{The man is blind.}

\end{CJK*}