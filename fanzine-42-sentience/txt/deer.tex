I'm growing tired from my age. My legs, which used to leap and bound, can no longer
even run for very long without giving out. They can walk still, with ease, but running has
become difficult. My hooves as well - they've become quite worrisome now; a bit
tarnished around the edges, and a pain has wrung across my front-right. It won't be long
now until even walking will be a pain. And the worst I've yet to mention: my neck. My
neck aches. My neck aches so very badly. It hurts to pick up food from the ground, and it
hurts to look around. It hurts to make sounds, and it hurts to place it down to rest. It's
the first and only pain I've had that refuses to leave.

I am sitting in the forest, letting my neck ease in the air. It's a rare, feverishly calm day
today. The clouds in the sky are greying a touch, but a fair amount of sky-blue and
sunlight-gold still pour through in patches. So vivid now, the stars that used to blanket
the night sky. How I loved them, their sparkle, and their glisten, only to be drowned out
by the beauty of the moon, with its own effervescent glow, and the enigma of its shape.
One day, I remember, as a fawn, I noticed that the moon was gone. The stars no longer
felt dim in comparison. They danced in the night sky, protected me and comforted me
as my mother left. Left to feed, to search, and eventually to die. I saw changing patterns
within their constellations, and I still wonder if those patterns have been recognised by
anyone else; have they lent their protection to others on the nights that I no longer
needed it? The next night, cold and alone, a small sliver of the moon returned.

Unlike the stars, the sun does not hold much grace in my eyes. I've always thought it
more a tool, to help find food and whatnot. How can something you can't look at be
considered beautiful? However, that doesn't mean I don't miss it. Most days now, the
sun doesn't come up. It stays behind the wretched clouds and emits naught but a faint
hue. That's why days like today are rare. Rays of warmth are peeking out from the
clouds, creating spots of true bliss where the encroaching cold cannot (and will not)
catch me. And, on days like today, I would say the sun is beautiful, even if I can't see it.
On days like today, I will take as much grace in the sun as I can, for my time is fleeting. I
can tell it may only be a few months, or weeks even. By the time all the leaves return to
their trees, I will join my mother.

The trees look empty like this, especially now, standing bare in the sunbeams. I always
get used to the umbrella of verdant green because that's what my mother loved, or at
the very least I think it is. And now it's winter, so the dead branches flake off bark bit by
bit, and the last clinging dead leaves will soon fall too. I hope that the birds and squirrels
are long gone by this point. No food is left here for them. I've thought it clever for a long
time to build a home in a tree. You get all the perks of the forest, without much fear of
predators in the night. I'm almost certain that they're gone and can only hope that
wherever they do go is safe. I'm no longer jolted up by the birds' flirting songs, or the
vacant stare of a squirrel expecting me to give them more food - irritating little bastards.
I love them too though, and they might never know that, but perhaps they don't need to.
They will return at some point, along with the green, and they'll run amok again. I
probably won't be here to see that. Maybe I should be saying good riddance; my final
moments will be in peace.

Peace is an odd word here. There is the near constant noise of the wind, or the birds, or
the nearby city to keep ``peace'' from being peaceful. The bugs, down on the ground, also
disturb me. When I was younger, my mind would never be able to rest knowing that the
bugs were crawling around, near me, all around me, and that the worms, which always
seemed to spawn from the ground just to make the pouring weather worse, burrowed
beneath me, whilst I was stuck on my own, without anybody to keep the creeps off my
fur. Do they know that they bother me? Do they know that they live, or do they just think
of how they can continue living without any awareness that they are alive?

It's beginning to get dark. It gets too dark too early now. It got too dark too early then. It
was the middle of winter, and my mother had gone out to find food. I was still too young
to be independent, so I stayed hidden away. Within the hour, the sun had gone down,
and it had become so dark that I could only see the stars in the sky. Hungry, I waited till
morning, however, by the time the sun came up, there was no trace of mother. The
prints had been lost to the shuffling of insects in the ground, and I assume she was too
far away to hear my calls. I was left alone. It was hard to find food, or to continue
moving. That was when I really started thinking about what I was doing. Where do I get
food? How do I continue living? And then, past that, I started thinking more deeply. Why
am I alive? What does it mean to be alive?

I won't ever be able to answer those questions, they're beyond me. And, in any case, I
need to start moving to find a place to sleep tonight. I've learnt that places too open
leave me susceptible to rain, and this place, with the half empty canopy, is far too open.
I've got to move. My weary joints, which take their time in raising my body from the
ground, become severed from my thoughts as the crick in my neck overwhelms me. My
only remaining thought is to move slowly and carefully, so as to not suddenly bring
about a harsh pain in my neck again. Eventually, after much morose effort, I am back at
my normal height above the ground, and I begin to walk.

I hate walking. It reminds me that I can't run anymore. It's a reminder that I'm not young
anymore. It's a reminder that I am alone. Each step carried the weight of memories I
would rather forget. Most of all, the one where I found my mother. After a long time
searching, and not enough time mourning, I came across a reindeer lying on the ground.
She was my mother, I was certain of it. Her antlers were gone, and her body was no
longer furry, but red, smooth and wet. She was dead. She was my mother, and she was
dead. I figured out after some time that facing me was the inside of her body. I came to
this conclusion since it hadn't rained, and even if it had, rain is not red. I figured that out
after all too long, and it sickened me. Pink and purple and red shapes spilled out
grotesquely from her deformed body, once alive. The antlers were cut off, cruelly, by
some maniac. Her legs were bent - frozen mid-flight. She was running at the instant she
died. She was running away from something, and it killed her anyway.

Moving through the woods now, I see it all. The gaps in my mind normally filled by the
squirrels and the birds, and even the bugs gnaw at my thoughts, insidiously staking their
claim of my quiet mind, which now rustles with the crunch of leaves and the snap of
twigs. And then, a smell. Unrecognisable at first, but familiar. Putrid and rotten, yet
homely. I can't help myself as I move towards it out of my sheer curiosity. And my legs
ache, they ache now for I have been walking for too long without rest, but the stench
pulls me ever closer. And finally, I see it. The spitting image of my mother, but not as dead.

It doesn't take long for me to realise that this isn't my mother. For one, my mother lost
her antlers and lost her fur; this reindeer has both. She is lying on the ground,
breathless. From that alone, I do not know if she is dead. She is younger than me, I know
that for sure. She also, strangely enough, has a hole in her neck. Small, and red. I
wonder if it is some sort of remedy for neck pain. Is it common in reindeer? The next
thing I notice, from there, is a pool of crimson flowing out on the frozen mud, dripping
down from the hole in her neck, staining her fur on the way down. I am then reminded of
the image of my mother, dead, and the rotten stench pouring from her neck. I never
understood until now - this is what killed her. Suddenly, the rustle of leaves picks up, not
from my movement but from a nearby bush. I turn to run away and a bang echoes
through my mind, as my empty brain notices that my neck no longer aches.